\chapter{State of the art}

\section{Current state of networks solution}

There's three main solution about blockchain network\footnote{https://blog.ethereum.org/2015/08/07/on-public-and-private-blockchains/}:

\begin{outline}
    \1 \textbf{Public blockchains}: a public blockchain is a blockchain that anyone in the world can read, anyone in the world can send transactions to and expect to see them included if they are valid, and anyone in the world can participate in the consensus process - the process for determining what blocks get added to the chain and what the current state is. As a substitute for centralized or quasi-centralized trust, public blockchains are secured by cryptoeconomics - the combination of economic incentives and cryptographic verification using mechanisms such as proof of work or proof of stake, following a general principle that the degree to which someone can have an influence in the consensus process is proportional to the quantity of economic resources that they can bring to bear. These blockchains are generally considered to be "fully decentralized".
    \1 \textbf{Consortium blockchains}: a consortium blockchain is a blockchain where the consensus process is controlled by a pre-selected set of nodes; for example, one might imagine a consortium of 15 financial institutions, each of which operates a node and of which 10 must sign every block in order for the block to be valid. The right to read the blockchain may be public, or restricted to the participants, and there are also hybrid routes such as the root hashes of the blocks being public together with an API that allows members of the public to make a limited number of queries and get back cryptographic proofs of some parts of the blockchain state. These blockchains may be considered "partially decentralized".
    \1 \textbf{Fully private blockchains}: a fully private blockchain is a blockchain where write permissions are kept centralized to one organization. Read permissions may be public or restricted to an arbitrary extent. Likely applications include database management, auditing, etc internal to a single company, and so public readability may not be necessary in many cases at all, though in other cases public auditability is desired.
\end{outline}

\subsection{Behind Blockchains}
The ideas behind the born of blockchain technology is to keep informations public and exploit cryptographic
algorithms to keep identity secret, mantaining authentication to perform the transaction process. That 
idea found a huge use cases application, in particular amoung the cryptocurrencies environment, that's 
the area in which the public blockchains found more applications. The main rule of the overall system is 
\textbf{"keep it transaparent, safe and anonymous"}, It mean that all the transactions processed by the 
public blockchain networks is \textbf{transaparent} and everyone has read access. The \textbf{Safe}
concept is the major application of the cryptographic science, for example the bitcoin wallet is based
on a key pair based on eliptic curve algorithms. \textbf{Anonymous} is strictly correlated to the Safe
concept, for the same reason that transactions are safe, the identity could be anonymous, if noone share
the identity associated to wallet public key. There's only a public key that correspond to the owner 
of a wallet with a balance inside in. 
\bigskip

On the other hand the majority of the companies, that would like to use blockchains solution for 
improvement on internal processes, don't want to share and keep own informations public. The goal
of the main Companies use cases is to handle internal company processes, such as supply chain process,
using a membership mechanism that allow some nodes of the governance to have read or write access to the 
ledger. 

\bigskip
To undestrand the behaviour about public or private blockchains, we are going to list the features
for both, in order to adapt the choice based of own needs: 

\begin{outline}[enumerate]
    \1 The main advantages of the \textbf{public blockchain} coul be fall into two major categories:
    \2 Public blockchains provide a way to protect the users of an application from the developers, 
    establishing that there are certain things that even the developers of an application have no 
    authority to do. The code is public and everyone could see how it works, on the other hand
    each identity of the user is protected by cryptographic algorithms.   
    \2 Public blockchains are open, and therefore are likely to be used by very many entities and gain 
    some network effects. To give a particular example, consider the case of domain name escrow. 
    Currently, if A wants to sell a domain to B, there is the standard counterparty risk problem that 
    needs to be resolved: if A sends first, B may not send the money, and if B sends first then A might 
    not send the domain.However, if we have a domain name system on a blockchain, and a currency on the 
    same blockchain, then we can cut costs to near-zero with a smart contract: A can send the domain to 
    a program which immediately sends it to the first person to send the program money, and the program 
    is trusted because it runs on a public blockchain.
    In other work public blockchain fully eliminate intermediary.

    \1 Compared to the public bloclchain the advantages of a \textbf{private blockchain} are:

    \2 The consortium or company running a private blockchain can easily, if desired, change the rules 
    of a blockchain, revert transactions, modify balances, etc. In some cases, eg. national land 
    registries, this functionality is necessary.
    
    \2 The validators are known, so any risk of a 51\% attack arising from some miner collusion in China 
    does not apply.

    \2 Transactions are cheaper, since they only need to be verified by a few nodes that can be trusted 
    to have very high processing power, and do not need to be verified by ten thousand laptops. This is 
    a hugely important concern right now, as public blockchains tend to have transaction fees exceeding 
    \$0.01 per tx.

    \2 Nodes can be trusted to be very well-connected, and faults can quickly be fixed by manual intervention, 
    allowing the use of consensus algorithms which offer finality after much shorter block times. 

    \2 If read permissions are restricted, private blockchains can provide a greater level of, well, privacy.
\end{outline}

From several analysis it get out that the 75\% of already implemented projects are designed 
specifically for private aim\footnote{\textbf{The State of the Art for Blockchain-EnabledSmart-Contract 
Applications in the Organization} - Chibuzor Udokwu∗, Aleksandr Kormiltsyn, Kondwani Thangalimodzi, 
Alex Norta - Department of Software Science Tallinn University of Technology, Tallinn, Estonia},
It means that is growing up the need to improve of the Consortium Blockchains that allow a
memberships mechanism build for company use cases, it maintain the transactions private in order 
to grant the privacy of the business process and data.
\\
On the other hand in some process is usefull to use public blockchain, so is growing up the need to
improve the interoperability about consortium and public blockchains into a crosschain solution.

{\renewcommand{\arraystretch}{2}%
\begin{tabular}{|l|l|l|l|l|}
  \hline
  \textbf{Blockchain Name} & \textbf{Network} & \textbf{Currency} & \textbf{Consensus} & \textbf{Smart Contract} \\
  \hline
  \textbf{Bitcoin} & Public & Bitcoin & PoW & Possible but less extendible\\
  \hline
  \textbf{Ethereum} & Public & Ether & PoS & Multiple programming languages\\
  ~ & ~ & ~ & ~ & (Solidity, Vyper)\\
  \hline
  \textbf{Hyperledger} & Permissioned - & None & Pluggable or & Multiple programming languages\\
  ~ & Federal/Private & ~ & PBFT & (Go, Java, Javascript, Solidity)\\ 
  \hline
  %\textbf{Eos} & No & Yes & Yes\\
  %\hline 
  %\label{table:compare-blockchains}
\end{tabular}}

\subsection{Overview}

Michael Burgess, chief operating officer of Ren said that \textbf{"All interoperability solutions will 
likely have trade-offs; so it's a matter of designing systems that find a balance between security, 
governance, adaptability, and economic incentives that suit their target market."}
\bigskip

\textbf{"Private chains operating without distributed consensus are more prone to data manipulation 
and the integrity of the data/assets being transferred from a private, permissioned and centralized 
chain to a more decentralized chain could be questioned. Overall, there is no one solution that fits 
all in terms of being public/private, centralized/decentralized — it is a broad spectrum with specific 
trade-offs."}\footnote{https://cointelegraph.com/news/blockchain-interoperability-the-holy-grail-for-cross-chain-deployment}
, quoting the words of Agarwal, CEO of Persistence.
\bigskip

The main idea of the industry experts is the trade-off concept to obtain a cross-chain interoperability,
between public and private. To understand it we are going to remind the strong differences in working 
foundamental of both blockchains:

\begin{outline}
    \1 \textbf{Public}: It's a peer to peer network in which every node is potentially untrusted, so
    the consensus mechanism is developed in order to prevent every malicous node that could compromise
    data and transactions performed over the network. The entire architecture and consesus are distributed
    in order to minimize the liability of data manipulation.

    \1 \textbf{Consortium}: The basis idea is that the network is composed by a set of truster or semi-trusted
    node, that compose the governance of the network. The consensus mechanism is not so stricted beacause 
    the starting ipothesis is different and usually it need to have a good performance and low latency of
    the transactions. On the other hand this features could lead to a data manipulation. 

\end{outline}

\subsection{Limitation}

Now that is clead the pros and cons of each networks the interoperability in some cases it could be 
the solution of many use cases problem, the public could allow an asset transactions amoung users without
limit and granted security and authentication, on the other hand consortium could allow a set of features
and informations that users and mostly companies don't want to keep it public. Nevertheless the 
Achilles heel is:


\begin{outline}
    \1 \textbf{Synchronization}: The both network must be synchronize and world state must be the same
    in each moment. It means that each transactions that involves both blockchains before to be archived
    the both networks must reach a strong synch amoung the ledgers.  

    \1 \textbf{Time Effort}: to make it usable it mean that the transactions and synchronization
    must be performed in a reasonable time.
    
    \1 \textbf{Identity}: over the blockchains network the identity if the user or in other word 
    the owner of the wallet is handled in several ways. For example Ethereum handle it as a Key pair,
    private and public, that allow authentication of the wallet owner, on the other hand Hyperledger
    Fabric the authentication mechanism for the user of the network is implemented with x.509 certificates.
    So an other problem is the mapping of that different mechanism that blockchains implements to allow
    authentications. 
\end{outline}


\subsection{CrossChain Current Solution}

This problem and new challenge of crosschain was born a few years ago and there was several solutions
that try to fix the problem and allow interoperability. The main ideas to perform interoperability is:

\begin{outline}
    \1 \textbf{New Blockchain}: During the year was born several networks and frameworks that propose
    to allow the interconnection between public and private blockchains. Many of that solutions
    are based on new blokchchain networks that are stuctured in order to allow architectural level
    the interoperability, one of that reality is Ark\footnote{\url{https://ark.io/}}. 
    But we know that one of the biggest challenge ramain to allow interconnection between the well-known 
    blockchain networks. 

    \1 \textbf{Architectural Framework}: There are thousand of frameworks proposed over the last years,
    but the cross-chain isn't still a reality. Nevertheless the main idea that shares the majority
    of proposed idea is the Sidechain. Introducing a new level between the major layer of the mainnet 
    that allow the mapping, using ad-hoc API, the request from one network to the other one. In a nutshell
    all the requests from public to private blockchains and vicevere pass by this Sidechain. 

    \1 \textbf{Atomic Swaps}: allow users to trade one cryptocurrency for another directly in a peer-to-peer 
    transaction Hashed TimeLock Contracts (HTLCs)\footnote{\url{https://en.bitcoin.it/w/index.php?title=Hash_Time_Locked_Contracts&source=post_page}}. Atomic swaps are not a true form of cross-chain 
    communication (as the two chains do not communicate), but a mechanism that allows two parties to 
    coordinate transactions across chains. Atomic swaps can be effective if used correctly and are the 
    mechanism that enables the Lightning Network\footnote{\url{https://lightning.network/}}.

    \1 \textbf{Relay}: llow a contract to verify block headers and events on another chain. Several 
    approaches to relays exist, ranging from verifying the entire history of a chain to verifying specific 
    headers on-demand. Each method has trade-offs between the cost of operation and the security of the relay. 
    Relays are often quite expensive to operate, as we saw first-hand with BTCRelay\footnote{\url{http://btcrelay.org/}}.

    \1 \textbf{Merged Consensus}: allow for two-way interoperability between chains through the use of a relay 
    chain. Merged consensus can be quite powerful, but generally must be built into the chain from the ground 
    up. Projects like Cosmos\footnote{\url{https://cosmos.network/}} and ETH2.0\footnote{\url{https://github.com/ethereum/eth2.0-specs}} use merged consensus.

    \1 \textbf{Federations}: allow a selected group of trusted parties to confirm the events of one chain on 
    another. While federations are powerful, their obvious limitation lies in the requirement to trust a 
    third party.

\end{outline}

\subsection{Chaincode EVM}

In our analysis we spend a great attention around Hyperledger and Ethereum, two of the main blockchain
solutions used in the world, the first one for permissioned use cases and the second one for public
processes. In the last year IBM technical ambassador developed an \textbf{EVM chaincode}\footnote{https://github.com/hyperledger/fabric-chaincode-evm} able to run
bytecode of Solidity smart contract over the Hyperledger Fabric network. It isn't a real cross-chain solution 
but it's a step forward interoperability amoung blockchains. Of course it still have many limit 
for the use, in particolar for the identity mapping from eth address to fabric identity and vice-versa. 


\section{Blockchain application in Fashion Environment}

\subsection{Provenance case & Martine Jarlgaard}

Thanks to the blckchain behaviour that allow an immutable record registered for each transaction 
performed over the supply chain of the items productions.
One of the first fashion house that start to use the blockchain technology for own company is 
Martine Jarlgaard that in 2017, made a partnership with Provenance\footnote{Provenance withepaper (2015) Blockchain: the solution for transparency in product supply chains. https://www.provenance.org/whitepaper} 
producing clothes with digital tag: The tag could be a QRCode od an RFID readed using NFC technology.
That tag provide the entire history of the related clothes, providing each step of the producing process.
\\
The actors of the supply chain process are:

\begin{outline}
    \1 \textbf{British Alpaca Fashion Farm}: It cares about alpacas livestock and shearing.
    \1 \textbf{Two Rivers Mill}: It cares about wool spinning.
    \1 \textbf{Knitster LDN}: It cares about the knitting process.
    \1 \textbf{Martine Jarlgaard}: It cares about design of the clothes an final work.
\end{outline}

Each actors of the supply chain is a blockchain node that take part at the supply chain pipe through
the transactions of the exchanged assets, such as wool, cloth and so on. each transaction is registered
over the blockchain and visible at each node. 

Customer side the user has a clear vision of the entire production process, from the material used to the 
item produced. It allow the company to gain in credibility and transparency of the products sell. 

\subsection{Counterfeiting - VeChain & BabyGhoast}


\section{Use Case asis}

Armadio Verde is an Italian community was born with the aim to share childrens clothes and than once it 
growing up, it allow to share adult clothes too. The working model is based on the share principle. Every 
user after is sign up to the platform could book a pick up of own old clothes. The clothes must be in a 
good state, clean and put in a box. Once the box arrive to Armadio Verde, the clothes going to be checked 
and evaluated, for each approved clothes is created a dedicated form with all the related information. 
After the upcycling process the clothes going to be shared over the platform store. The user that send 
the clothes earn an amount of "star"(the money used over the platform). The star coul be used to purchase 
other clothes adding a few euro for each item. The clothes that couldn't share over the platform for 
reselling process, is sent to a certified Onlus.

\subsection{Sustainability Token}

\subsubsection{PlasticToken}

It's an ERC20 chaincode that run over the hyperledger fabric network\footnote{https://ptwist.eu/}. It provides functionalities to read
and write, with access and rights control, into the distributed ledger. The ERC20 chaincode is the software 
handling the PlasticTokens in a secure manner. These tokens are up to the ERC20 standard, meaning a fixed 
amount of tokens will be minted when the chaincode is deployed. This amount is called “TotalSupply”, and 
will be assigned to a special user, called “centralbank” in the current implementation.
Once the original PlasticToken supply is minted, users can interact with it via a “transfer” functionality. It
allows the central bank to send tokens to any previously enrolled user, then each user can use this same
function to transfer tokens between each other

It run over the Plastic Twist project. 

\subsubsection{ECOCoin}

The ECO coin is a new cryptocurrency that is earned through sustainable action. The ECO coin
aims to reward anyone, anywhere in the world carrying out sustainable actions. Eating meat-free
meals, switching to a green energy provider or riding a bike to work can earn you ECOs which
user could spend in ECO new sustainable marketplace to buy ecological experiences, services and
goods. 

It's based on consortium blockchain architecture anche each marketplace that want to involve their
business in ECO environment it must be accepted as governance member of the network. 