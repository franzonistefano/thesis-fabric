\documentclass[
    % classica,
	% numerazioneromana,
	tipotesi = magistrale,
	12pt
]{toptesi}

\usepackage[T1]{fontenc}
\usepackage[utf8]{inputenc}
\usepackage{amsmath}
\usepackage[sorting=none]{biblatex}
\usepackage{booktabs}
\usepackage{enumerate}
\usepackage{float}
\usepackage{lipsum}	% TODO Eliminare
\usepackage{listings}
\usepackage{hyperref}
\usepackage{svg}
\usepackage{url}
\usepackage{xspace}
\usepackage{xcolor}
\usepackage{outlines}

\usepackage{multirow}
\usepackage{tabularx}
\usepackage{makecell}
\usepackage{spverbatim}
\usepackage{graphicx}
\usepackage{flafter} % make sure figures do not appear before their text
\input{solidity-highlighting.tex}

\hypersetup{
	pdfpagemode = {UseOutlines},
	bookmarksopen,
	pdfstartview = {FitH},
	colorlinks,
	linkcolor = blue,
	citecolor = blue,
	urlcolor = blue,
	% Info
	pdftex,
	pdfauthor = {Stefano Franzoni},
	pdftitle = {Crosschain with Hyperledger and Ethereum},
	% pdfsubject = {},
	% pdfkeywords = {Politecnico di Torino, Italy}
}

%%% Paragraph settings
\setlength{\parindent}{0pt}
\setlength{\parskip}{1ex plus 0.5ex minus 0.2ex}


%%% Custom commands
% HTTP request methods
\newcommand{\GET}{\texttt{GET}\xspace}
\newcommand{\POST}{\texttt{POST}\xspace}
\newcommand{\PUT}{\texttt{PUT}\xspace}

\addbibresource{bibliography.bib}

\begin{document}

\english

\frontmatter
% \pagenumbering{roman}

\begin{ThesisTitlePage}
	\candidato{Stefano \textsc{Franzoni}}
	\relatore{ing.\ Valentina \textsc{Gatteschi}} 
	\secondorelatore{prof.\ Fabrizio \textsc{Lamberti}}
    \tutoreaziendale{dott.\ Alfredo \textsc{Favenza}}
	\ateneo{Politecnico di Torino}
	\titolo{Blockchain and smart contracts in the Fashion industry}
	\sottotitolo{A Decentralized Application built on Ethereum and Hyperledger Fabric}
	\corsodilaurea{Ingegneria Informatica (Software Engineering)}
    \sedutadilaurea{\textsc{Anno accademico} 2019 -- 2020}
	\logosede{logopolito}
\end{ThesisTitlePage}

\frontmatter
\pagenumbering{roman}

\ringraziamenti % acknowledgements
ACKNOWLEDGMENTS

\vspace*{5\baselineskip}

%\include{abstract}
%\include{ringraziamenti}

\tableofcontents
\listoffigures
\listoftables

\mainmatter
\pagenumbering{arabic}
\chapter{Introduction}

Over the last few years, the Blockchain technology has been living an era of growth, many different applications for 
it being found in several different fields. The first application of blockchain was in finance with the Bitcoin, right
after many environments such as logistics, law and administration found huge applications in it. 
\bigskip

This kind of technology is based on a peer to peer network that allows to transfer assets among users involved in the 
network, without the need for a third party. The overall network and the transactions, performed by the joint nodes, 
are handled by a consensus algorithm that performs the validation of the transactions and updates the ledger by adding 
the transactions block to the chain. The ledger is a file containing a set of records, each record is a transaction 
processed by the network. Moreover, all the nodes involved in the network share the same ledger that contains the state 
of the overall network. The most significant use of the blockchain technology is in the finance environment, which founds 
a great instrument in cryptocurrencies, exploiting the decentralized architecture, the security advantages and the 
transparency. In fact, everyone has read access over the network and he can visualize the data of the transactions processed.
\bigskip

In the last few years, many companies have been getting interested in the blockchain applications. Since, in many cases, 
the data that they share are sensitive information and companies don't want to keep them public, the companies need for 
a solution a little bit different. Therefore, recently, this interest has resulted in the first permissioned blockchain 
solutions, maintained by a consortium system of nodes.
\bigskip

In fact, the blockchain world nowadays are split between permissioned and permissionless blockchains:
\begin{outline}
    \1 \textbf{Permissionless Blockchain}: The permissionless blockchain is a fully decentralized network where anyone in the world can read or send transactions, as well as participate in the consensus process.

    \1 \textbf{Permissioned Blockchain}: the permissioned solution is a blockchain where the consensus process is controlled by a 
    pre-selected set of nodes, which could be defined as a "partially decentralized" network. Moreover, a membership mechanism could 
    be implemented, in order to to handle the read and write access over the network. 
\end{outline}

Based on the above considerations, the two main objectives of this thesis work are:
\begin{outline}
    \1 \textbf{Blockchain and supply chain management}: The goal is to create a blockchain solution for the 
    management of the supply chain process of a Fashion Company. In particular, the thesis addresses the 
    issue of transparent fashion upcycling, which is generally characterized by processes involving many 
    different actors. The goal of the blockchain-based solution devised is to track the items over the flow 
    more clearly and transparently as possible.

    \1 \textbf{Cross-chain solution}: The goal is to implement a cross-chain interaction between Hyperledger 
    Fabric and Ethereum network. Fabric manages all the aspects related to the supply chain (orders, 
    production, part of sale process). Ethereum instead is only used to the end-user sell process of 
    the items. The goal is to reach the interoperability between public and private blockchains. 
\end{outline}

\bigskip

The target reached at the end of the developed thesis work would conceivably be:

\begin{outline}
    \1 \textbf{Simple management process}: It is to reach a simplified model of the overall management process of the 
    transactions and users involved into the system. 

    \1 \textbf{Supply Chain increased transparency}: A simplified handling process of the supply chain. The goal is to make as
    clear as possible the tracking process of the clothes over the actors involved, from the producing of the items
    to the selling process. 

    \1 \textbf{Technology improvements and modular solutions}: A cross-chain technology means to generalize a 
    solution that could be applied to other use cases. The integration among more blockchains networks,
    is a big challenge nowadays, in the following chapter I described in details the current situation and 
    the solution chosen. 
\end{outline}

\bigskip

The rest of the document is structured as follows:
\begin{outline}
    \1 \textbf{Chapter 2 - State of the art}: focuses on previous studies in this field. It gives an overview of the 
    current solutions to the issue faced in the thesis work. 

    \1 \textbf{Chapter 3 - Solution}: explains how the problem at hand was solved using the method and theory. It 
    shows all the technologies used to implement the chosen solution and how they interact with each other describing 
    the implementation details. 

    \1 \textbf{Chapter 4 - Results}: describes the outcome of the tests done. It is tested different subprocess.

    \1 \textbf{Chapter 5 - Conclusion}: handles the conclusion, I summarize the overall conclusions.
    
\end{outline}
\chapter{State of the art}

This chapter gives an overview of the current solutions and technologies used during the thesis work.

\section{What is Blockchain}

In the previous chapter, I briefly described the Blockchain behaviors. Blockchain is a growing list of records, called 
blocks, that are linked. Each block contains a cryptographic hash of the previous block, a timestamp, and transaction 
data.\cite{blockchain-definition}

By design, a blockchain is immutable and resistant to data modification. It runs over a peer to peer network and each
node of the network maintains a copy of the \textit{distributed ledger}. Therefore, exploiting the peer to peer network 
architecture and the cryptographic science, it is obtained a distributed data storage, immutable, secure, and continuously 
synchronized among network nodes. The other core part of the Blockchain is the \textit{consensus} mechanism, it is the 
algorithm that handles the consensus process in charge of validate transactions and add it to the chain of blocks, 
performing the ledger update.

\subsection{Consensus mechanism}

The main consensus algorithms are:
\begin{outline}
    \1 \textbf{Byzantine fault tolerance - BFT}: The concept of Byzantine Fault Tolerance in Blockchain is the 
    feature of reaching an agreement or consensus about particular blocks based on the proof of work, even when some 
    nodes are failing to respond or giving out malicious values to misguide the network. The main objective of BFT is 
    to safeguard the system even when there are some faulty nodes. This will also help to reduce the influence of faulty 
    nodes.\cite{bft}
    \2 \textbf{Proof of work - PoW}: it deters \textit{denial-of-service} attacks. A proof of work is a piece of data 
    which is difficult to produce but easy for others to verify and which satisfies certain requirements. Producing a 
    proof of work can be a random process with low probability so that a lot of trial and error is required on average 
    before a valid proof of work is generated. In other words, it is like a problem to solve spending a lot of computing 
    power to validate transactions and create new blocks.\cite{pow}
    
    

    \1 \textbf{Proof of stake - PoS}: it uses a pseudo-random election process to select a node to be the validator of 
    the next block, based on a combination of factors that could include the staking age, randomization, and the node’s 
    wealth. The size of the stake determines the chances for a node to be selected as the next validator to forge the 
    next block - the bigger the stake, the bigger the chances. Where in Proof of Work-based systems more and more 
    cryptocurrency is created as rewards for miners, the Proof-of-Stake system usually uses transaction fees as a 
    reward.\cite{pos}
\end{outline}

Usually, over the network, there are specific nodes, called \textit{miners} with huge computational power, which handles
the transactions in exchange for \textit{transaction fees} and they have a reward for each block created. 

Figure \ref{fig:pow-work} shows an example of how the PoW algorithm works. 
\begin{figure}[h!]
    \centering
    \includegraphics[totalheight=5cm]{img/puzzle.PNG}
    \caption{How PoW works}
    \label{fig:pow-work}
\end{figure}

\subsection{51\% attack}
It can be performed when a group of miners controlling more than 50\% of the network's mining hash rate or computing power.
The attackers would be able to prevent new transactions from gaining confirmations, allowing them to halt payments between 
some or all users
It bring to a network's fork. Figure \ref{fig:attack} shows the attack situation, considering that in the blockchain is 
kept just the longer branch, meanwhile, the shorter one is discarded.\cite{attack}
 
\begin{figure}[h!]
    \centering
    \includegraphics[totalheight=5cm]{img/51attack.jpg}
    \caption{51\% attack}
    \label{fig:attack}
\end{figure}

Nevertheless, the 51\% attack does not pay, because it needs a huge computational power to be performed, and once the 
network is compromised, no one continues to use it. It brings everybody to play by the rules. 

\subsection{Bitcoin}
It is a digital currency and there is no central bank controlling it. It is created by an unknown person or group of 
people under the name \textit{Satoshi Nakamoto}. At the end of 2008, he releases the whitepaper, a manifesto where 
Satoshi explained all the technology behind Bitcoin, introducing the Blockchain. In 2009 he releases the open-source 
software and Bitcoin become reality. 

\subsection{Smart Contract}

The smart contract concept was introduced by computer scientist \textit{Nick Szabo}. It is a computer program or a 
transaction protocol which is intended to automatically execute, control or document legally relevant events and 
actions according to the terms of a contract or an agreement.\cite{sc-def}. Since the integration of the smart
contract inside Blockchain, it is possible to exploit the network to transfer each kind of asset described by the 
contract. 


\section{Current state of networks solution}

Starting from the Introduction's consideration, below I listed the details of the three main solutions for 
blockchain networks\cite{private-public}:

\begin{outline}
    \1 \textbf{Public blockchains}: they are peer to peer networks in which anyone in the world has read 
    access to the network, anyone can send transactions over the network, and anyone in the world can 
    participate in the consensus process. In a public blockchain every node is potentially 
    untrusted, so the consensus mechanism is developed in order to prevent every malicious node that could 
    compromise data and transactions performed over the network. The entire architecture and consensus are 
    distributed in order to minimize the liability of data manipulation. The consensus process defines the blocks that get added to the chain
    determining the current state of the network. The security issue is solved by a mix of cryptographic algorithms
    and cryptoeconomics solution. The idea is to combine economic incentives proportional to the resources that
    the node can bring to bear. The resources targeted depend on the consensus algorithm used. For example proof of work(PoW)
    involves computational resources into the consensus mechanism. On the other hand, proof of stake(PoS) involves
    the token amount of the node involved in consensus. These blockchains are generally considered 
    to be "fully decentralized".
    \1 \textbf{Consortium blockchains}: The basic idea of the consortium blockchain is that the network is 
    composed by a set of trusted or semi-trusted nodes, that compose the governance of the network.
    The consensus mechanism is not as complex as that of public blockchains, because the starting hypothesis is 
    different and usually it needs to have a good performance and low latency of the transactions.
    I provide an example to understand how governance's nodes are involved in the consensus process: 
    One might imagine a consortium of 15 financial institutions, each of them operates a node and 10 of them 
    must sign every block in order for the block to be valid. The right to read the blockchain may be public, 
    or restricted to the participants, and there are also hybrid routes such as the root hashes of the blocks 
    being public together with an API that allows members of the public to make a limited number of queries 
    and get back cryptographic proofs of some parts of the blockchain state. These blockchains may be 
    considered "partially decentralized".
    \2 \textbf{Fully private blockchains}: a fully private blockchain is a consortium blockchain where write permissions 
    are kept centralized to just one organization. Read permissions may be public or restricted to an arbitrary 
    extent. Likely applications include database management, auditing, etc internal to a single company. Therefore, 
    public readability may not be necessary at all in many cases, although, in other cases public auditability 
    is desired.
\end{outline}

\subsection{Behind the Blockchains}
With the blockchain technology, cryptographic science found the most applications. The concept behind 
the public blockchain is that all the transaction data are completely public; nevertheless, the identity of the 
user involved is kept secret. This idea has found a huge application in the cryptocurrencies environment. 
The \textit{Fintech}\footnote{It is the technology and innovation that aims to compete with traditional financial methods in the delivery of financial services.} 
is the environment in which the public blockchains has found the most applications. The main rule of the overall system is 
\textbf{"keep it transparent , safe and anonymous"}, it means that all the transactions processed by the 
public blockchain networks are \textbf{transparent } and every node has read access. The \textbf{Safe}
concept is granted by the combination of cryptographic science and economic incentives. \textbf{Anonymous}
has granted thanks to the cryptographic science applications, for example, the bitcoin wallet is based on a key pair 
computed on elliptic curve algorithms. If no one shares the identity associated with the wallet public key, or keep 
the private key public, the user identity continues to be anonymous. 
\bigskip

As explained above the blockchain world is divided between public and private blockchain.  
To understand the behavior about public or private blockchains, we are going to list the features
for both, in order to adapt the choice based on own needs: 

\begin{outline}[enumerate]
    \1 The main advantages of the \textbf{Public blockchain} could fall into two major categories:
    \2 Public blockchains provide a way to protect the users of an application from the developers; 
    the code is public and everyone can see how it works. This solution limits the authority 
    of the developers over the application. Moreover, the user identity is always mapped into a wallet address.   
    \2 Public blockchains are open, and therefore are likely to be used by many entities and gain 
    some network effects. Besides, the public blockchain fully eliminates intermediaries. Here is an example 
    of a transfer of ownership case. A wants to sell an item to B. Right now there is a standard risk 
    problem of the involved counterparty: if A sends first, B may not send money, and if B sends first the 
    money A might not send the item. All the problems related to these kinds of cases could be resolved using 
    smart contracts, running over the public blockchain, moreover, the costs is close to zero. 
    With the smart contract implementations, A can send the item, to be sold, to a program that immediately 
    sends it to the first person that in the meanwhile sends money to the program. 
    
    \1 Compared to the public blockchain, the advantages of a \textbf{Private blockchain} are:

    \2 The consortium or companies running a private blockchain can easily, if desired, change the rules 
    of a blockchain, revert transactions, modify balances, etc. In some cases, eg. national land 
    registries, this functionality is necessary.
    
    \2 The validators are known, so any risk of a 51\% attack\footnote{ It is a potential attack on a blockchain network, where a single entity or organization is able to control the majority of the hash rate, potentially causing a network disruption.\url{https://academy.binance.com/security/what-is-a-51-percent-attack}}, arising from some miner collusion in China, 
    does not apply.

    \2 Transactions are cheaper, since they only need to be verified by a few nodes that can be trusted 
    to have very high processing power, and do not need to be verified by ten thousand laptops. This is 
    a hugely important concern right now, as public blockchains tend to have transaction fees exceeding 
    \$0.01 per tx.

    \2 Nodes can be trusted to be very well-connected, and faults can quickly be fixed by manual intervention, 
    allowing the use of consensus algorithms which offer finality after much shorter block times. 

    \2 If read permissions are restricted, private blockchains can provide a greater level of, well, privacy.
\end{outline}

From many analysis it gets out that the 75\% of already implemented projects are designed specifically for 
private aim \cite{blockchain-state},
which means that a need is growing to improve the Consortium Blockchains that allow a memberships 
mechanism build for company use cases, which maintains the transactions private to guarantee the privacy of 
the business process and data.
\\
On the other hand, in some processes it is useful to implements public blockchain solutions, so there is a growing need to 
improve the interoperability about a consortium and public blockchains into a cross-chain solution.

%{\renewcommand{\arraystretch}{2}%
%\begin{tabular}{|l|l|l|l|l|}
%  \hline
%  \textbf{Blockchain Name} & \textbf{Network} & \textbf{Currency} & \textbf{Consensus} & \textbf{Smart Contract} \\
%  \hline
%  \textbf{Bitcoin} & Public & Bitcoin & PoW & Possible but less extendible\\
%  \hline
%  \textbf{Ethereum} & Public & Ether & PoS & Multiple programming languages\\
%   ~ & ~ & ~ & ~ & (Solidity, Vyper)\\
%   \hline
%   \textbf{Hyperledger} & Permissioned - & None & Pluggable or & Multiple programming languages\\
%  ~ & Federal/Private & ~ & PBFT & (Go, Java, Javascript, Solidity)\\ 
%  \hline
  %\textbf{Eos} & No & Yes & Yes\\
  %\hline 
  %\label{table:compare-blockchains}
%\end{tabular}}

%\begin{table}[htbp]
%    \centering
%    \begin{tabular}{|c|c|c|c|c|p{1cm}p{1cm}p{1cm}p{1cm}p{1cm}p{1cm}p{1cm}|}
%    \hline
%    \textbf{Blockchain Name} & \textbf{Network} & \textbf{Currency} & \textbf{Consensus} & \textbf{Smart Contract}  \\ \hline
%    \textbf{Bitcoin} & Public & Bitcoin & PoW & Possible but less extendible \hline
%    \multirow{ 2}{*}{1} \textbf{Ethereum} & Public & Ether & PoS & Multiple programming languages \\
%    ~ & ~ & ~ & ~ & (Solidity, Vyper)\hline
%    \multirow{ 2}{*}{1} \textbf{Hyperledger} & Permissioned - & None & Pluggable or & Multiple programming languages \\
%    ~ & Federal/Private & ~ & PBFT & (Go, Java, Javascript, Solidity)\hline
%    \end{tabular}
%    \caption{A test caption}
%    \label{table1}
%\end{table}

\bigskip

Table \ref{table-compare-blockchains} shows a comparing among three of the main blockchain networks. The comparison
is based on the kind of the \textbf{network}, the \textbf{currency} that runs over each network, the \textbf{consensus} 
algorithm used and the possibility to develop and run \textbf{smart contract} over that. 

\begin{table}[h]
    {\renewcommand\arraystretch{1.25}
    \begin{tabular}{|l|l|l|l|l|} \hline
    \textbf{Name} & \textbf{Network} & \textbf{Currency} & \textbf{Consensus} & \textbf{Smart Contract}\\ \hline\hline
    \textbf{Bitcoin} & Public & Bitcoin & PoW & Possible but less extendible\\ \hline
    \textbf{Ethereum} & Public & Ether & PoS & Solidity, Vyper\\ \hline
    \textbf{Hyperledger} & Permissioned & None & PBFT & Go, Java, Javascript, Solidity\\ \hline
    \end{tabular}}
    \caption{Comparing among blockchains features}
    \label{table-compare-blockchains}
\end{table}

\subsection{CrossChain and interoperability}

Michael Burgess, chief operating officer of Ren states that \textbf{"All interoperability solutions will 
likely have trade-offs; so it's a matter of designing systems that find a balance between security, 
governance, adaptability, and economic incentives that suit their target market."}
\bigskip

\textbf{"Private chains operating without distributed consensus are more prone to data manipulation 
and the integrity of the data/assets being transferred from a private, permissioned and centralized 
chain to a more decentralized chain could be questioned. Overall, there is no one solution that fits 
all in terms of being public/private, centralized/decentralized — it is a broad spectrum with specific 
trade-offs."}\cite{interoperability1}
, quoting the words of Agarwal, CEO of Persistence.
\bigskip

What is getting from the point of view of the industry experts; it is a trade-off solution to obtain cross-chain 
interoperability, between public and private.

\subsubsection{Limitations}

Considering the pros and cons of each network listed, interoperability could, in some cases, be the 
solution of many cases problem, for example, the public blockchain could allow an asset transaction among 
users without limit and granted security and authentication. On the other hand, consortium solution could 
allow companies to set up roles over their own network and to keep information data private. Nevertheless, 
the integration between the two blockchain solutions has several Achilles heels to be evaluated and managed:


\begin{outline}
    \1 \textbf{Synchronization}: both networks must be synchronized and the world state must be the same in 
    each moment. This means that each transaction that involves both blockchains, must reach strong synch among 
    the ledgers, before being validated. 

    \1 \textbf{Time}: in order for it to be usable, the transactions and synchronization
    must be performed in a reasonable time.
    
    \1 \textbf{Identity}: each blockchain implementation handles the identity mechanism in a specific way. 
    This means that the user wallet is implemented using specific cryptographic algorithms and solutions. For 
    example, Ethereum handles it as a Key pair, private and public, that allow authentication of the wallet 
    owner. On the other hand, Hyperledger Fabric implements the authentication mechanism for the user of the 
    network using x.509 certificates. So the other problem is the mapping of these different mechanisms that 
    blockchains implement to allow authentications. 
\end{outline}


\subsubsection{Current solution}

The new challenge of cross-chain was born a few years ago and it has brought many companies and research centers 
to design solutions to fix the problem and allow interoperability. \textbf{Figure \ref{fig:crosschain-interaction}} 
shows the theoretical solution to the interoperability problem between Bitcoin and Ethereum, at each layer of 
blockchain architecture\cite{crosschain-level}.

\begin{figure}[h!]
	\centering
	\includegraphics[totalheight=6cm]{img/crosschain.PNG}
	\caption{CrossChain Interactions}
	\label{fig:crosschain-interaction}
\end{figure}

The main ideas to perform interoperability is:

\begin{outline}
    \1 \textbf{New Blockchain}: Over the last few years several networks and frameworks have appeared that propose to 
    allow the interconnection between public and private blockchains. Many of those solutions are based on new 
    blockchain networks that are structured in order to allow, architectural level, the interoperability, for 
    example, Ark\cite{ark} is a blockchain-based platform that allows anyone to customize their own blockchain. But 
    the biggest challenges still remains to allow interconnection between the well-known blockchain 
    networks. 

    \1 \textbf{Architectural Framework}: There are thousands of frameworks proposed over the last few years, but 
    the cross-chain isn't still a consolidated reality. Nevertheless, most of the solutions share the same idea, a 
    Sidechain\cite{sidechain} between the two blockchains. Introducing a new layer between the two mainnet that allows 
    mapping, using ad-hoc API, the requests from one network to the other one. In a nutshell all the requests 
    from the one to the other blockchain and vice-versa passing by the sidechain.\cite{atomic-crosschain} \cite{iot-crosschain}

    \1 \textbf{Atomic Swaps}\cite{atomic-swap}: it allows users to trade one cryptocurrencies for another directly in a peer-to-peer 
    transaction Hashed TimeLock Contracts (HTLCs)\cite{HTLC}. Atomic swaps are not a true form of cross-chain 
    communication (as the two chains do not communicate), but a mechanism that allows two parties to 
    coordinate transactions across chains. Atomic swaps can be effective if used correctly and are they are the 
    mechanism that enables the Lightning Network\cite{lightning}.

    \1 \textbf{Relay}: it allows a contract to verify block headers and events on another chain. Several 
    approaches to relays exist, ranging from verifying the entire history of a chain to verifying specific 
    headers on-demand. Each method has trade-offs between the cost of operation and the security of the relay. 
    Relays are often quite expensive to operate, as we saw first-hand with BTCRelay\cite{relay}.

    \1 \textbf{Merged Consensus}: it allows for two-way interoperability between chains through the use of a relay 
    chain. Merged consensus can be quite powerful, but generally must be built into the chain from the ground 
    up. Projects like Cosmos\cite{cosmos} and ETH2.0\cite{eth2} 
    use merged consensus.

    \1 \textbf{Federations}: it allows a selected group of trusted parties to confirm the events of one chain on 
    another. While federations are powerful, their obvious limitation lies in the requirement to trust a 
    third party.

    \1 \textbf{Chaincode EVM}: In the last year, IBM technical ambassador developed an \textbf{EVM chaincode}\cite{evm-chaincode} 
    able to run bytecode of Solidity smart contract over the Hyperledger Fabric network. It is not a real 
    cross-chain solution but it is a step forward interoperability among blockchains. It still has many limits, 
    for example, there is not a real identity mapping mechanism from eth address to fabric identity and vice-versa.

\end{outline}

In the thesis work, I focused my attention on Hyperledger and Ethereum, two of the main blockchain 
solutions used in the world, the former for permissioned cases, the latter one for public processes. For that reason 
I choose to exploit the \textit{Chaincode EVM} to implements the thesis work solution. 
 

\section{Blockchain application in Fashion Environment}

In the last few years, many Fashion companies have been getting interested in the blockchain for tracking and 
counterfeiting issues.

\subsection{Provenance case & Martine Jarlgaard}

Thanks to the blockchain feature it is possible to store in an immutable way the record associated with each 
transaction performed over the supply chain.  One of the first fashion houses that started to use the blockchain 
technology for its own company is Martine Jarlgaard that in 2017, the fashion company made a partnership with Provenance\cite{provenance} 
producing clothes with digital tag: The tag could be a QRCode or an RFID reader using NFC technology.
That tag provides the entire history of the related clothes, providing each step of the producing process.
\bigskip

The actors of the supply chain process are:

\begin{outline}
    \1 \textbf{British Alpaca Fashion Farm}: It cares about alpacas livestock and shearing.
    \1 \textbf{Two Rivers Mill}: It cares about wool spinning.
    \1 \textbf{Knitster LDN}: It cares about the knitting process.
    \1 \textbf{Martine Jarlgaard}: It cares about the design of the clothes and the final work.
\end{outline}

Each actor of the supply chain is a blockchain node that takes part in the supply chain pipe through 
the transactions of the exchanged assets, such as wool, cloth, and so on. Each transaction is registered over the 
blockchain and visible at each node. 

Customer side the user has a clear vision of the entire production process, from the material used to the 
item produced. It allows the company to gain credibility and transparency of the products sale. 

\subsection{Counterfeiting - VeChain & BabyGhoast}

BabyGhoast by combining blockchain technology with NFC chips, it creates a digital identity for each cloth 
produced. It improved the tracking process over the supply chain. Moreover, it allows protecting the brand 
and the users against counterfeit items. In order to implement the solution it is used VeChain technology, that includes 
inside BabyGhoast clothes an RFID/NFC chip or QRCode, that allows identification of the item thanks to a 
unique ID VeChain. Moreover, by scanning the chip or QRCode using the VeChain Pro application, it is possible to 
access the data related to the item and the production process. 

\section{ASIS model}

Armadio Verde is an Italian community that was born to share children's clothes. Once it grew up, it allows 
adult clothes sharing too. The working model is based on the sharing principle. Every user, after is signing up 
to the platform, can book a pick up of their old clothes. The clothes must be in a good state, clean and 
put in a box. Once the box arrives at Armadio Verde, the clothes are going to be checked and evaluated. For each 
approved clothes, a dedicated form  is created with all the related information. After the upcycling process, 
the clothes are shared over the platform store. The user that sent the clothes earns an amount of 
"star"(the money used over the platform). The star could be used to purchase other clothes adding a few euros 
for each item. The clothes that could not be shared on the platform for the reselling process, are sent to a 
certified Onlus.

\section{Sustainability Token}

\subsubsection{PlasticToken}

Plastic Token is an ERC20 chaincode that runs over the Hyperledger Fabric network\cite{plastic-coin}. It provides functionalities to read
and write, with access and rights control, into the distributed ledger. The ERC20 chaincode is the software 
securely handling the PlasticTokens. These tokens are up to the ERC20 standard, meaning a fixed 
amount of tokens will be minted when the chaincode is deployed. This amount is called “TotalSupply” and 
will be assigned to a special user, called “central bank” in the current implementation.
Once the original PlasticToken\cite{ptwist} supply is minted, users can interact with it via a “transfer” functionality. It
allows the central bank to send tokens to any previously enrolled user, then each user can use this same
function to transfer tokens between each other

It runs over the Plastic Twist project. 

\subsubsection{ECOCoin}

The ECO coin\cite{eco-coin} is a new cryptocurrency that is earned through sustainable action. The ECO coin
aims to reward anyone, anywhere in the world carrying out sustainable actions. Eating meat-free
meals, switching to a green energy provider or riding a bike to work can earn you ECOs which
users could spend in ECO new sustainable marketplace to buy ecological experiences, services and
goods. 

It is based on consortium blockchain architecture and each marketplace that want to involve their 
business in ECO environment must be accepted as a governance member of the network. 

\subsection{Solution}

As explained before, the Fashion environment has several advantages to exploit transparency and traceability of the 
blockchain solution. Nevertheless, right now the current solutions are based on consortium blockchains, that allow 
handling the internal process in a better way, but it is less useful if applied to the end-user. For that reason, in 
the thesis work, I developed the interoperability between the two networks. The internal processes are still managed 
by the consortium blockchain, exploiting all the membership's advantages and maintaining transparency. On the other 
hand parts of the end-user side, are handled by using a public blockchain, in order to detach as much as possible by own 
case, the token involved, considering the token used as a reference asset and not just for own use. Moreover, there 
is no concept for the interoperability implementation between Hyperledger Fabric and Ethereum blockchains, as specified 
there are many architectural differences between the two networks. Therefore, in the thesis work, I proposed an API 
based solution, at Application layer, that performs the cross-chain between the two networks.
\chapter{

\section{Oveview} 

\subsection{Actors}

The overview and flow of the Dapp developed is shown in \textbf{Figure 1}. 
The \textbf{Actors} involved in the system are:

\begin{outline}
    \1 \textbf{User}: It's the \textit{end user}. It use the web-app to send old clothes and purchase from Reclothes store
    \1 \textbf{Reclothes Admin}: It's the \textit{system admin}, it perform the actions in order to handle the system
    \1 \textbf{Producer}: It's part of the upcycling process. It receive the materials to perform the recycling process
\end{outline}

Each of that access to the system wich different permission and priviledges. Once the user is logged in, 
It could access to several features. It's possible to split the overview flow
into 2 subflow starting from Reclothes actor, the \textbf{User side} and the \textbf{Producer side}.

\begin{outline}[enumerate]
	\1 \textbf{User Side}
    \2 User send Box with old clothes and receive Fabric points and ERC20 Token
    \2 User purchase items inside dapp store using Fabric points and ERC20 Token
    \1 \textbf{Producer Side}
    \2 Reclothes send clothes box with old matherials and receive Regeneration Credits
    \2 Reclothes spend the Regeneration Credits to purchase upcycled clothes by Producer
\end{outline}

\subsection{Token echanged}

The \textbf{Token} exchanged over the networks will be:

\begin{outline}[enumerate]
	\1 \textbf{Over Fabric Net}
    \2 \textbf{User Token}: It's a point used to handle part of payment system, points based, related to clothes
    shipping from User to Reclothes and viceversa.
    \2 \textbf{Regeneration Credits}: It's a point used to handle part of credit system, points based, related to clothes
    shipping from Reclothes to Producers and viceversa.

    \1 \textbf{Over Ethereum Net}
    \2 \textbf{CO2 Token}: It's and ERC20 Token run over public network in charge to handle part of payments
    related to clothes shipping from User to Reclothes and viceversa.
\end{outline}


\begin{figure}[h!]
	\centering
	\includegraphics[totalheight=6cm]{img/use-case-schema.png}
	\caption{UseCase Overview}
	\label{fig:schema}
\end{figure}


\section{Technologies used to perform crosschain interaction}

\subsection{Tool Used}

\begin{outline}
    \1 \textbf{Metamask}: It's used as ethereum wallet to perform and sign the transactions started by dapp 
    \1 \textbf{Web3}: It's the software library used to interact with smart contract
    \1 \textbf{Fab3 Proxy}: It map the Web3 API with the Fabric SDK in order to interact with
    Fabric network. It perform a mapping between the Fabric Identity (X.509) with an eth address, generated on the fly,
    used to perform dapp call. 
    \1 \textbf{Fabric Chaincode EVM}: It's the EVM chaincode that allow to run Solidity smart contract
    over the Fabric network
    \1 \textbf{Expressjs}: Web Framework used to develop web-app and smart contract API 
    \1 \textbf{Infura}: allow to run a Ethereum node in order to set an endpoint used to interact with own contract.
    \1 \textbf{Docker}: The fabric network components run inside Docker containers.
\end{outline}

Figure ~\ref{architecturalFlow} show The Architectural Flow and how the technologies is used and interact each other.

\begin{figure}[h!]
	\centering
	\includegraphics[totalheight=13cm]{img/architectural_flow.png}
	\caption{Architectural Flow}
    \label{architecturalFlow}
\end{figure}
 
\section{Use Cases}

\subsection{UseCase 1 - User Side}

As shown in \textbf{Figure 2} both Actors User and Reclothes Admin, once is logged in, access to a set
of features. The Use case diagram show all the action that both users could perform over the networks and
tha flows that each actions follow. The features are split over the two network, the Fabric one and the 
Ethereum one. 

The Internal Flow of the \textit{\bf{Send Box}} macro process is the follow one:

\begin{outline}[enumerate]
    \1 User send box with old clothes
    \1 Reclothes Admin receive box, evaluate it
    \1 The web app perform the payments from Reclothes Account to User Account
    \1 Once both transactions succed, both token are accredited and User could spend it
\end{outline}

\subsubsection{Transactions}

The two process that start the transactions are the \textit{\bf{Evaluation}} performed by The Reclothes Admin
and the \textit{\bf{Purchase Items}} performed by the Users over the platform store. 

\\

\begin{outline}[enumerate]
    \1 The \textit{\bf{Evaluation}} process works in the following way:
    \2 Reclothes Admin visualize the next pending request to be evaluated
    \2 Reclothes Admin evaluate it and set an amount value of Fabric points and ERC20 Token to be send
    \3 The points are sent over Fabric network invoking a chaincode function
    \3 The ERC20 Token are send over Ethereum network (Ropsten), sending the token to the ETH Address stored
        in the smart contract during the User Registration Phase.
    \2 Once both transactions succed, both tokens are accredited and User could spend it

    \1 The \textit{\bf{Purchase Items}} process in a similar way:
    \2 User choose the items to purchase over the web-app store
    \2 Reclothes Admin evaluate it and set a value amount of Fabric points and ERC20 Token to send
    \3 The points are sent over Fabric network invoking a chaincode function
    \3 The ERC20 Token are send over Ethereum network (Ropsten), sending the token to the ETH Address stored
        in the smart contract during the User Registration Phase.
    \2 Once both transactions succed, both token are accredited and User could spend it
\end{outline}

\begin{figure}[h!]
	\centering
	\includegraphics[totalheight=15cm]{img/use_case1.png}
	\caption{UseCase 1}
	\label{fig:usecase1}
\end{figure}


\subsection{UseCase 2 - Producer Side}

The Use case diagram shown in \textbf{Figure 3} describe how Reclothes Admin and Producer interact each other.
In this case all the features are performed over the Hyperledger Fabric network.

\newline \\
We could split the flow into two subflow, the first one from Reclothes to Producer, that we could identify
with the two actions \textit{Send Box} and \textit{Purchase Box}, on the other hand the second subflow
from Producer to Reclothes is summarize into \textit{Evaluate Material} function. 

\newline \\
All the asset exchange is handle using \textbf{Regeneration Credits} a Fabric token exchanged
and handle by the Fabric chaincode and running over Fabric network.

\\
\begin{outline}[enumerate]
    \1 \textbf{from Reclothes to Producer}   
    \2 \textbf{Send Box}
    \3 Send Box with old materials to be recycled by the Producer Company
    \3 Receive \textbf{Regeneration Credits} based on the old materials evaluation
    
    \2 \textbf{Purchase Box}
    \3 Reclothes Admin purchase Box by Producer Company relized with recycled materials, spending the
    Regeneration Credits
    
    \1 \textbf{from Producer to Reclothes}
    \2 \textbf{Evaluate Material}
    \3 Producer Admin perform the evaluation of the materials received by Reclothes, after the evaluation
    the transactions of Regeneration Credits is performed by the Fabric chaincode
\end{outline}


\begin{figure}[h!]
	\centering
	\includegraphics[totalheight=10cm]{img/use_case2.png}
	\caption{UseCase 2}
	\label{fig:usecase2}
\end{figure}


\section{Smart Contract}

All the smart contract are developed in Solidity. 

Both the Fabric contracts are use the \texttt{fabric-chaincode-evm} in order to allow Solidity code into fabric network

\begin{outline}[enumerate]
    \1 \textbf{Hyperledger Fabric}
    \2 \textbf{User Contract}: handle the the User side, registration and interactions phase.
    \2 \textbf{Producer Contract}: handle the interaction from Reclothes to Producers.
    
    \1 \textbf{Ethereum}
    \2 \textbf{ERC20 Contract}: it's a standard smart contract with a Max Supply fixed to 100.000.000 .
    The Contract run over Ropsten network and we access to it through the Infura node.
\end{outline}

\subsection{User Contract} 

\paragraph{Data Structure}
The model of the data structures is divided inside 4 structs:

\begin{outline}[enumerate]
    \1 \textbf{User}: model all users data
    \1 \textbf{Admin}: model Reclothes Admin data
    \1 \textbf{PointsTransaction}: Model transactions data and incorporate \texttt{TransactionType} used to identify the flows
    \1 \textbf{ClothesBox}: The box sent with old clothes 
\end{outline}

\begin{lstlisting}[language=Solidity]
        // model a user
        struct User {
            address userAddress;    // User address (inside fabric environment)
            address publicAddress;  // external eth public address of User Admin
            string firstName;
            string lastName;
            string email;
            uint points;            // Fabric points amount
            bool isRegistered;      // Flag for internal use
            uint numTransaction;    // number of transactions performed
            mapping(uint => PointsTransaction) userTransactions;
            uint numBox;            // number of box transaction evaluated
            mapping(uint => ClothesBox) box;
        }
    
        // model a admin
        struct Admin {
            address adminAddress;   // Admin address (inside fabric environment)
            address publicAddress;  // external eth public address of Admin
            string name;
            bool isRegistered;      // Flag for internal use
        }
    
        // model points transaction
        enum TransactionType { Earned, Redeemed }
        struct PointsTransaction {
            uint points;
            TransactionType transactionType;
            address userAddress;    // user address involved
            address adminAddress;   // admin address involved
        }
    
        // model clothes box to ship
        struct ClothesBox {
            address userAddress; // reclothes-producer Admin
            uint tshirt;        // Number of item
            uint pants;         // Number of item
            uint jacket;        // Number of item
            uint other;         // Number of item
            bool isEvaluated;   // Flag to check if box evaluation is performed
            uint points;        // fabric value amount of the box
        }
\end{lstlisting}

\paragraph{Transactions}

There's two functions that performs transaction from and to Reclothes:

\begin{outline}[enumerate]
    \1 \textbf{earnPoints}: It's an internal function called by \texttt{sendBox} function. 
    It performs the fabric points transaction from Reclothes to User.
    \1 \textbf{usePoints}: performs the fabric points transaction when the User purchase item by 
    Reclothes store.
\end{outline}

\begin{lstlisting}[language=Solidity]
    //update users with points earned
    function earnPoints (uint _points, address _userAddress ) onlyAdmin(msg.sender) internal {

      // verify user address
      require(users[_userAddress].isRegistered, "User address not found");

      // update user account
      users[_userAddress].points = users[_userAddress].points + _points;

      PointsTransaction memory earnTx = PointsTransaction({
        points: _points,
        transactionType: TransactionType.Earned,
        userAddress: _userAddress,
        adminAddress: admins[msg.sender].adminAddress
      });

      // add transction
      transactionsInfo.push(earnTx);

      users[_userAddress].userTransactions[users[_userAddress].numTransaction] = earnTx;
      users[_userAddress].numTransaction++;

      usersTransactions[totTx] = earnTx;
      totTx++;

    }


    //Update users with points used
    function usePoints (uint _points) onlyUser(msg.sender) public {

      // verify enough points for user
      require(users[msg.sender].points >= _points, "Insufficient points");

      // update user account
      users[msg.sender].points = users[msg.sender].points - _points;

      PointsTransaction memory spendTx = PointsTransaction({
        points: _points,
        transactionType: TransactionType.Redeemed,
        userAddress: users[msg.sender].userAddress,
        adminAddress: 0
      });

      // add transction
      transactionsInfo.push(spendTx);

      users[msg.sender].userTransactions[users[msg.sender].numTransaction] = spendTx;
      users[msg.sender].numTransaction++;

      usersTransactions[totTx] = spendTx;
      totTx++;
    }
\end{lstlisting}


\subsection{Producer Contract}

\paragraph{Data Structure}
The model of the data structures is divided inside 3 structs:

\begin{outline}[enumerate]
    \1 \textbf{Producer}: model all producers data
    \1 \textbf{Admin}: model Reclothes Admin data
    \1 \textbf{ClothesBox}: The box sent with old clothes 
\end{outline}

\begin{lstlisting}[language=Solidity]
    // model a producer
    struct Producer {
        address adminAddress;   // Producer Admin address (inside fabric environment)
        address publicAddress;  // external eth public address of Producer Admin
        string name;            // Producer admin name
        bool isRegistered;      // Flag for internal use
        uint numBox;            // number of box transactions evaluated
        uint pointsProvided;    // amount of points provided by own evaluations
        mapping(uint => ClothesBox) box;
    }

    // model a admin
    struct Admin {
        address adminAddress;   // Admin address (inside fabric environment)
        address publicAddress;  // external eth public address of Admin
        string name;            // Admin name
        bool isRegistered;      // Flag for internal use
        uint numBox;            // number of box transaction evaluated
        uint creditSpent;       // amount of points provided by own evaluations
        mapping(uint => ClothesBox) box;
    }

    struct ClothesBox {
        address adminAddress; // reclothes-producer Admin
        uint tshirt;        // Number of item
        uint pants;         // Number of item
        uint jacket;        // Number of item
        uint other;         // Number of item
        bool isEvaluated;   // Flag to check if box evaluation is performed
        uint points;        // fabric value amount of the box

        //mapping(uint => Clothes) clothes;
    }
\end{lstlisting}

\paragraph{Transactions}

There's two functions that perform transactions from and to Reclothes:

\begin{outline}[enumerate]
    \1 \textbf{evaluateBox}: function called by Producer to evaluate box materials, sent by Reclothes, 
    to send Regeneration Credits.
    \1 \textbf{buyUpcycledBox}: function called by Reclothes Admin that spend earned Regeneration Credits
    to purchase upcycled clothes by Producer. 
\end{outline}

\begin{lstlisting}[language=Solidity]
    // Evaluate Old Box
    function evaluateBox(uint _points) onlyProducer() public {
        //check correct pending request index
        require(evaluatedIndex < pendingIndex, "No more pending request");

        //check if evaluation is done
        require(!pendingBox[evaluatedIndex].isEvaluated, "Request just evaluated");

        //pop pending request
        ClothesBox storage box = pendingBox[evaluatedIndex];

        //update box transaction
        box.isEvaluated = true;
        box.points = _points;

        //add evaluated box
        evaluatedBox[evaluatedIndex] = box;
        evaluatedIndex++;

        debtPoints += _points;
        totPointsProvided += _points;
    }

    function buyUpcycledBox(uint _tshirt, uint _pants, uint _jackets, uint _other, uint _points) onlyAdmin() public {
        require(debtPoints >= _points, "Not enought credits accumulated in old material boxes");

        ClothesBox memory box = ClothesBox({
         adminAddress: msg.sender,
         tshirt: _tshirt,
         pants: _pants,
         jacket: _jackets,
         other: _other,
         isEvaluated: true,
         points: _points
        });

        admins[msg.sender].box[admins[msg.sender].numBox] = box;
        admins[msg.sender].numBox++;
        admins[msg.sender].creditSpent += _points;

        //add upcycled box
        upCycledBox[upCycledIndex] = box;
        upCycledIndex++;

        debtPoints -= _points;
        totBoxNew++;
    }

\end{lstlisting}

\subsection{ERC20 Contract}

It's a standard ERC20 token with the following features:

\begin{outline}
    \1 \textbf{Symbol}: CO2
    \1 \textbf{Name}: CarbonToken
    \1 \textbf{Total supply}: 100000000
    \1 \textbf{Decimals}: 18
\end{outline}

The main features of the contract are describer by ERC20 interface

\begin{lstlisting}[language=Solidity]
    contract ERC20Interface {
        function totalSupply() public constant returns (uint);
        function balanceOf(address tokenOwner) public constant returns (uint balance);
        function allowance(address tokenOwner, address spender) public constant returns (uint remaining);
        function transfer(address to, uint tokens) public returns (bool success);
        function approve(address spender, uint tokens) public returns (bool success);
        function transferFrom(address from, address to, uint tokens) public returns (bool success);
    
        event Transfer(address indexed from, address indexed to, uint tokens);
        event Approval(address indexed tokenOwner, address indexed spender, uint tokens);
    }
\end{lstlisting}

\section{Network Architecture}

The network Architecture build for the application includes:

\begin{outline}[enumerate]
    \1 \textbf{1 Orderer} organization with \textit{1 ordered} running

    \1 \textbf{3 Organizations} each with 1 peer, Peer0, running
    \2 \textit{Org1}: User Organization
    \2 \textit{Org2}: Reclothes Admin Organization 
    \2 \textit{Org3}: Producer Organization

    \1 \textbf{2 Channels}
    \2 \textit{Chanel12}: It's the cannel between Org1 and Org2 and allow the comunication between User and Reclothes
    \2 \textit{Chanel23}: It's the cannel between Org2 and Org3 and allow the comunication between Reclothes and Producer
\end{outline}

\begin{figure}[h!]
	\centering
	\includegraphics[totalheight=8cm]{img/fabric_network.png}
	\caption{Fabric Network}
	\label{fig:fabric_network}
\end{figure}

\subsection{Fabric Network}

The Figure show how components interact each other. We could separate components into 2 categories,
inside and outside Fabric Network. First of all we need to describe the components involved :

\begin{outline}
    \1 \textbf{Web3 App}: It's the Dapp and the Client connection to the network
    \1 \textbf{Channel}: It's the channel above which transfert data 
    \1 \textbf{CA}: It's the Certification Authority in charge of release certificates.
    \1 \textbf{Peer}: It's "Fabric node", the endpoint of the internal network. It own by specific CA with 
    fixed permissions, linked to the connected channels. 
    \1 \textbf{evm SC}: It's the Ethereum Virtual Machine Chaicode, used to run Solidity Smart Contract. The chaincode 
    is installed over the peer.
    \1 \textbf{ledger}: It's the ledger associated to the channel connected. There's a 1 to 1 association 
    between ledger and channel.
    \1 \textbf{CC}: It's the \textit{Consortium}, It's associated to the channel, manage ownerships and 
    It include a set of Organizations allowed. 
    \1 \textbf{Docker}: The network components run inside docker containers. 
\end{outline}

\begin{figure}[h!]
	\centering
	\includegraphics[totalheight=10cm]{img/network.png}
	\caption{Fabric Network Components}
	\label{fig:network}
\end{figure}

The \textit{chaincode} it's invoked calling the evm chaincode by the \textit{App} Client, using the channel communication.
Than the chaincode installed over the peer once is invoked agreed to the request and invoke the chaincode("smart contract")
method. Once the method returns, the chaincode forward the reply to the App client. The Dapp forward the
answer to the \textit{Orderer} peer that validate the response, create a new block, add it to the chain,
communicate it to the peer in order to syncronize the network and updating the Ledger World State.

\subsubsection{Config File}

To design and set up network components and rules, It's wrote the \texttt{config.yaml} file.
The network is structured in the following lines of code:

\begin{lstlisting}
    Organizations:
    - &OrdererOrg
        Name: OrdererOrg
        ID: OrdererMSP
        MSPDir: crypto-config/ordererOrganizations/example.com/msp
        Policies:
            Readers:
                Type: Signature
                Rule: "OR('OrdererMSP.member')"
            Writers:
                Type: Signature
                Rule: "OR('OrdererMSP.member')"
            Admins:
                Type: Signature
                Rule: "OR('OrdererMSP.admin')"

    - &Org1
        Name: Org1MSP
        ID: Org1MSP
        MSPDir: crypto-config/peerOrganizations/org1.example.com/msp
        Policies:
            Readers:
                Type: Signature
                Rule: "OR('Org1MSP.admin', 'Org1MSP.peer', 'Org1MSP.client')"
            Writers:
                Type: Signature
                Rule: "OR('Org1MSP.admin', 'Org1MSP.client')"
            Admins:
                Type: Signature
                Rule: "OR('Org1MSP.admin')"
        AnchorPeers:
            - Host: peer0.org1.example.com
              Port: 7051

    - &Org2
        Name: Org2MSP
        ID: Org2MSP
        MSPDir: crypto-config/peerOrganizations/org2.example.com/msp
        Policies:
            Readers:
                Type: Signature
                Rule: "OR('Org2MSP.admin', 'Org2MSP.peer', 'Org2MSP.client')"
            Writers:
                Type: Signature
                Rule: "OR('Org2MSP.admin', 'Org2MSP.client')"
            Admins:
                Type: Signature
                Rule: "OR('Org2MSP.admin')"
        AnchorPeers:
            - Host: peer0.org2.example.com
              Port: 8051

    - &Org3
        Name: Org3MSP
        ID: Org3MSP
        MSPDir: crypto-config/peerOrganizations/org3.example.com/msp
        Policies:
            Readers:
                Type: Signature
                Rule: "OR('Org3MSP.admin', 'Org3MSP.peer', 'Org3MSP.client')"
            Writers:
                Type: Signature
                Rule: "OR('Org3MSP.admin', 'Org3MSP.client')"
            Admins:
                Type: Signature
                Rule: "OR('Org3MSP.admin')"
        AnchorPeers:
            - Host: peer0.org3.example.com
              Port: 9051


                                        ...
                                        ...
                                        ...

Profiles:

    OrdererGenesis:
        <<: *ChannelDefaults
        Orderer:
            <<: *OrdererDefaults
            Organizations:
                - *OrdererOrg
            Capabilities:
                <<: *OrdererCapabilities
        Consortiums:
            SampleConsortium:
                Organizations:
                    - *Org1
                    - *Org2
                    - *Org3

    Channel12:
        Consortium: SampleConsortium
        <<: *ChannelDefaults
        Application:
            <<: *ApplicationDefaults
            Organizations:
                - *Org1
                - *Org2
            Capabilities:
                <<: *ApplicationCapabilities
    Channel23:
        Consortium: SampleConsortium
        <<: *ChannelDefaults
        Application:
            <<: *ApplicationDefaults
            Organizations:
                - *Org2
                - *Org3
            Capabilities:
                <<: *ApplicationCapabilities    
\end{lstlisting}

    \subsubsection{End to End Interactions} 
    Going deeper, the Figure show the flow of the end to end communication. How all the components are boxed 
    inside the Peer component. The Fab3 map the web3 request and forward it to fabric peer. the request 
    arrive to the evmcc that invoke Solidity smart contract methods.

    \begin{figure}[h!]
        \centering
        \includegraphics[totalheight=7.5cm]{img/EndToEnd.png}
        \caption{End To End}
        \label{fig:end_to_end}
    \end{figure}

    \subsubsection{Chaincode Invocations}
    The Figure below describe the internal workflow of the chaincode invocation, where's involved the 
    \textit{Client} the \textit{Peer} and the \textit{Orderer}. All the information are transfer over the 
    setted channel and in our study case, the client doesn't interact directly, but using \textit{Fab3 Proxy}
    as intermediary.

    \begin{figure}[h!]
        \centering
        \includegraphics[totalheight=6cm]{img/sc_invokation.png}
        \caption{Smart Contract Invocation Process}
        \label{fig:sc_invokation}
    \end{figure}

\subsection{Ethereum Network - Ropsten}

To run the ERC20 token it's used the testnet Ropstan against the mainnet. 
To set up and upload own ERC20 Token over the ethereum network it's used:

\begin{outline}
    \1 \textbf{My Ether Wallet}: To upload ERC20 contract 
    \1 \textbf{Etherscan.io}: To monitor and analyze transactions over the network
    \1 \textbf{Metamask}: To create user wallets
    \1 \textbf{Infura}: To set up a node in order to use it as endpoint and communicate with the Ropsten network,
    it is used as \textit{Provider} in \textit{Web3} library.  
\end{outline}

\section{Dapp - Client}

\subsection{Thecnologies used}

To develop the client application is used the following Technologies:

\begin{outline}
    \1 \textbf{Expressjs}: It's a node.js framework that allow to develop api for own application
    \1 \textbf{Bootstrap}: To build a user friendly front-end in order to interact in the best way
    \1 \textbf{Web3}: Ethereum Javascript API, It's is a collection of libraries that allow you to interact with a local or remote ethereum node 
    \2 \textbf{web3 0.20.2}: used for dapp developments, fabric side, It's a stable version and it's the version
    used in \texttt{fabric-chaincode-evm} development
    \2 \textbf{web3 1.0.0}: used for ethereum transactions, It's a version with more functionalities but less stable.
\end{outline}

Starting from the Homepage the User is allowed to register itself as \textbf{User}, \textbf{Reclothes Admin}
or \textbf{Producer}.

\subsection{Core part of the web-app}

The technical files and flow that dapp follow to run up it's the following one.

\begin{outline}[enumerate]
    \1 \textbf{Contract Address Generation}:
    \2 This step is in charge to run a script that deploy the contract adresses to be reffered during the 
    app running.
    \3 \textbf{UserContract.js}: running the script using node command, it return the address of the deployed contract
    \3 \textbf{ProducerContract.js}: running the script using node command, it return the address of the deployed contract
    \1  \textbf{dapp.js}: there's the core file that handle the contracts invokations, set up the contract address
    referance, and connect to a specific Fab3 instance.
    \1 \textbf{app.js}: It set up the API called by the web-app, map the request and forward to \texttt{dapp.js}.
\end{outline}

\subsection{Views}

\subsubsection{Homepage}

The homepage allow user to view the feature of each User type and to access to the registration page.

\begin{figure}[h!]
    \centering
    \includegraphics[totalheight=7.5cm]{img/dapp/home1.png}
    \caption{Home}
    \label{fig:home}
\end{figure}

\begin{figure}[h!]
    \centering
    \includegraphics[totalheight=7.5cm]{img/dapp/home2.png}
    \caption{Registration Phase}
    \label{fig:registration}
\end{figure}

\subsubsection{User Page}

The User page allow to view an overview infos once the user is logged in. 

\begin{enumerate}[-]
    \item \textbf{Address}: It's the public eth address setup during registration phase. 
    \item \textbf{Points Balance}: It's the Fabric points balance earned by the user sending the boxes.
    \item \textbf{ERC20 Balance}: It's the eth balance of the public token running over eth network.
\end{enumerate}

The Figure xxxx show how to compile the form in order to send box with old clothes. It's a simulation
of the real process to sending box, that in the real case could be implemented thow a QRCode or RFID
placed over the boxes.

The Figure xxx show how should be the store, purchase items over the platform start the transaction 
process. 

There's other section about infos that user is allowed to see. \textbf{Transactions} performed over
the fabric network and \textbf{Box Requests} that's all the history about the box sent and received
with all the related informations. 

\begin{figure}[h!]
    \centering
    \includegraphics[totalheight=7.5cm]{img/dapp/user-info.png}
    \caption{User Info}
    \label{fig:user_info}
\end{figure}

\begin{figure}[h!]
    \centering
    \includegraphics[totalheight=7.5cm]{img/dapp/user-send.png}
    \caption{Send Box}
    \label{fig:send_box}
\end{figure}

\begin{figure}[h!]
    \centering
    \includegraphics[totalheight=7.5cm]{img/dapp/user-buy.png}
    \caption{Purchase Clothes}
    \label{fig:purchase_clothes}
\end{figure}

\subsubsection{Reclothes Admin Page}

In the previous sections we talk about a logical split about Admin for User and Admin for Producers.
In the following views we divide the feature releated to the User type to be handled and there's a 
streight distinctions about Users and Producers.

\paragraph{Admin For Users}

This section show the view of the Admins that handle User side.

\begin{figure}[h!]
    \centering
    \includegraphics[totalheight=7.5cm]{img/dapp/admin-info.png}
    \caption{Admin Info}
    \label{fig:admin_info}
\end{figure}

\begin{figure}[h!]
    \centering
    \includegraphics[totalheight=7.5cm]{img/dapp/admin-evaluate.png}
    \caption{Evaluate Box}
    \label{fig:evaluate_box}
\end{figure}

\begin{figure}[h!]
    \centering
    \includegraphics[totalheight=7.5cm]{img/dapp/admin-tx.png}
    \caption{Transactions}
    \label{fig:admin-tx}
\end{figure}

\paragraph{Admin For Producers}

This section show the view of the Admins that handle Producer side.

\begin{figure}[h!]
    \centering
    \includegraphics[totalheight=7.5cm]{img/dapp/adminp-info.png}
    \caption{Admin for Producers Info}
    \label{fig:adminp-info}
\end{figure}

\subsubsection{Producer}

This section show the view of the Producer side.

\begin{figure}[h!]
    \centering
    \includegraphics[totalheight=7.5cm]{img/dapp/producer-info.png}
    \caption{Producers Info}
    \label{fig:producer-info}
\end{figure}

}
\chapter{Results}

\section{Target archived}

The goals to archive include both logical and technical target. 
The improvements reached by thesis development are the following ones:

\newline
The main target to reach is to improve Value Chain \footnote{This process includes the following phases: design and 
development of the product,raw materials management, production, shipping, selling and final use} value of the 
overall system.
\newline
The goal could be split inside \textbf{Technical Goal} and \textbf{Logical Goal}.

\begin{outline}
    \1 \textbf{Technical Goal}
    \2 \textbf{CrossChain Interaction}: Integrates into the same application both permissioned and permissionless 
    Blockchain networks. The integration is done application side. Some API endpoints start transactions in both 
    networks, one over Fabric network and the other one over Ethereum.
    \2 \textbf{Traceability}: This goal is archived implementing smart contracts, Hyperledger Fabric side,
    that tracks all the clothes box and store the entire transactions passed over the system. 
    
    \1 \textbf{Logical Goal}
    \2 \textbf{Supply Chain}: The target is to simplify the supply chain process, all the steps inside the chain
    are handled as transactions, stored over the ledger and updating world state and smart contract data. 
    \2 \textbf{Sustainability}: The entire process aims to support sustainability. Thanks to traceability feature,
    it is possible to follow the lifetime of the clothes until they finish to Producer, that performs the 
    material recycling in order to produce new upcycled clothes.   
    \2 \textbf{Counterfeiting}: Assign a UID to each clothes produced it is possible to fight the Counterfeiting 
    implementing new features such as the clothes registrations. In that way it is possible to have a secure register 
    containing all the clothes.
\end{outline}

\section{Use Case Test}

\subsection{Use Case 1 - Unit Test 1}

\subsubsection{Send Box and Evaluation}

Performing the Test over the Use Case 1 about the send box and evaluation processes. The following
figures show the results over the call of the related methods and how the application works. 

\textbf{Figure \ref{fig:user-send-box}} shows the log when User performs the \textit{Send Box} action.

\begin{figure}[h!]
	\centering
    \includegraphics[totalheight=4cm]{img/test/test1/user-send-box.png}
	\caption{User Send Box}
	\label{fig:user-send-box}
\end{figure}

Once the Box Request was successfully sent, the smart contract is invoked and the transaction is 
performed. Admin could visualize the pending box requests to be evaluated. Then the Reclothes Admin, UI side,
insert the value amount of the tokens and start the evaluation process.
\newline
\textbf{Figures \ref{fig:init-tx-reclothes-user}} and \textbf{\ref{fig:tx-reclothes-user}} 
shows the Fabric Transaction performed then the initialization of the Ethereum transaction. In the end, once 
the eth transaction was performed, the etherscan link associated with the related TransactionHash, allow the 
user to see the transaction history and info.

\begin{figure}[h!]
	\centering
    \includegraphics[totalheight=4cm]{img/test/test1/init-tx-RU.png}
	\caption{Init Transaction from Reclothes to User}
	\label{fig:init-tx-reclothes-user}
\end{figure}

\begin{figure}[h!]
	\centering
    \includegraphics[totalheight=3cm]{img/test/test1/performed-tx-RU.png}
	\caption{Init Transaction from Reclothes to User}
	\label{fig:tx-reclothes-user}
\end{figure}

Figure \ref{fig:fabric-tx} and Figure \ref{fig:eth-tx} shows the proof of the transactions succeed. 
The first Figure shows the User page that allows visualizing the history of transactions done and all related 
requests. The second one shows the etherscan page with all the information about Ethereum's transaction, in 
this case from Reclothes eth Account to User Account.

\begin{figure}[h!]
	\centering
    \includegraphics[totalheight=7cm]{img/test/test1/fabrix-tx.png}
	\caption{Fabric transaction history}
	\label{fig:fabric-tx}
\end{figure}

\begin{figure}[h!]
	\centering
    \includegraphics[totalheight=7cm]{img/test/test1/etherscan.png}
	\caption{Ethereum transaction over etherscan}
	\label{fig:eth-tx}
\end{figure}

\subsection{Use Case 1 - Unit Test 2}

\subsubsection{Purchase Item}

The \textbf{Figures \ref{fig:tx-user-reclothes}} shows the Purchase process. As the figure shows
there is, first of all, the Fabric transaction and then the Eth transaction. Once all
the previous checks are performed, the transaction from User account to Reclothes account starts. Once the 
transaction is performed the method prints the etherscan link to monitor the transaction and
all the related info. 

\begin{figure}[h!]
	\centering
    \includegraphics[totalheight=7cm]{img/test/test2/tx-user-reclothes.png}
	\caption{Transaction from User to Reclothes}
	\label{fig:tx-user-reclothes}
\end{figure}


\subsection{Use Case 2 - Unit Test 1}

To test the Use Case 2 I decided to track the behaviors of two processes:

\begin{outline}
    \1 \textbf{Send Old Material and Evaluation}: The Admin for Producer sends a box with inside the old materials
    to be recycled. Once the box arrived at Producer, then it is going to be evaluated and starts a 
    Regeneration Credits transaction from Producer to AdminP.

    \1 \textbf{Purchase Upcycled Material}: The Admin for Producer spends the earned Regeneration Credits
    to purchase by Producer recycled materials. The purchase options right now are three:
    \2 \textbf{Small Box}: 50 Regeneration Credits for 5 upcycled items.
    \2 \textbf{Medium Box}: 150 Regeneration Credits for 15 upcycled items.
    \2 \textbf{Big Box}: 200 Regeneration Credits for 40 upcycled items.
\end{outline}

\subsubsection{Send Old Material and Evaluation}

The \textbf{Figures \ref{fig:send-old-clothes}} shows the log of the send old clothes process.
In that case is sent a box with inside: 
\begin{outline}
    \1 \textbf{t-shirt}: 10
    \1 \textbf{pants}: 20
    \1 \textbf{jacket}: 10
    \1 \textbf{other}: 10
\end{outline}

\begin{figure}[h!]
	\centering
    \includegraphics[totalheight=6cm]{img/test/usecase2/0-send-old-clothes.png}
	\caption{Admin Send old clothes}
	\label{fig:send-old-clothes}
\end{figure}

Once the box is sent, the evaluation process starts. The Producer evaluates materials received
and issue Regeneration Credits amount that Admins could spend when need, to purchase
recycled items. The \textbf{Figure \ref{fig:1-box-tobe-evaluate.png}} shows the page used
to perform evaluation Process by Producer. In that case I set a Regeneration Credits amount
of 1200. The \textbf{Figure \ref{fig:evaluation-old-clothes}} shows the output of the evaluation process.
\\
Once the evaluation process is archived and the Regeneration Credits is sent
from Producer to Admin. The \textbf{Figure \ref{fig:evaluation-old-clothes}} shows the info update
of the Admin for Producer.

\begin{figure}[h!]
	\centering
    \includegraphics[totalheight=7cm]{img/test/usecase2/1-box-tobe-evaluate.png}
	\caption{Box to be evaluated}
	\label{fig:box-tobe-evaluate}
\end{figure}

\begin{figure}[h!]
	\centering
    \includegraphics[totalheight=4cm]{img/test/usecase2/2-evaluation.png}
	\caption{Producer Evaluate old materials}
	\label{fig:evaluation-old-clothes}
\end{figure}

\begin{figure}[h!]
	\centering
    \includegraphics[totalheight=4cm]{img/test/usecase2/3-credits-received.png}
	\caption{Admin Info update}
	\label{fig:credits-received}
\end{figure}

\subsubsection{Purchase Upcycled Material}
 
Once the Admin sent the box with old clothes and the evaluation process is archived, Admin has a 
Regeneration Credits amount to spend purchasing upcycled clothes by Producer and then resell them
inside the platform store. In our test case, we purchase a \textbf{Middle Box} spending an amount 
of 150 Regeneration Credits.
\\
The \textbf{Figure \ref{fig:buy-recycled-clothes}} shows the output of the purchase process and the 
Tot Regeneration Credits update once the purchase box process is performed.
\\
The \textbf{Figure \ref{fig:producer-infos}} shows the update of Producer Information,
the circulating Regeneration Credits amount is changed and the Tot Box New number is updated.

\begin{figure}[h!]
	\centering
    \includegraphics[totalheight=5cm]{img/test/usecase2/4-buy-recycled-clothes.png}
	\caption{Purchase Recycled Clothes}
	\label{fig:buy-recycled-clothes}
\end{figure}

\begin{figure}[h!]
	\centering
    \includegraphics[totalheight=5cm]{img/test/usecase2/5-producer-info-update.png}
	\caption{Producer Infos Update}
	\label{fig:producer-infos}
\end{figure}
\chapter{Conclusion}

For a clearer understanding and overview of all the work produced, I listed each part developed below:

\begin{outline}
    \1 \textbf{Hyperledger Network}: the network architecture is designed based on case needs. It is 
    implemented and produced with a set of scripts that automatize all the setting up and running processes.
    \1 \textbf{Smart Contracts}: two smart contracts, that shape the use case deal, are produced.
    \1 \textbf{ERC20 Token}: an ERC20 token in running and accessible over the Ropsten network, is produced.
    \1 \textbf{Dapp}: a web-application that integrates both the blockchain used, is produced.
\end{outline}

Therefore below I listed the targets archived:

\begin{outline}
    \1 \textbf{Simple managements process}: the process is structured in a simplified way, clear and easy
    to use and manage. Each actor is associated with a specific organization inside the network, taking part 
    in the network's governance with the specific right access.

    \1 \textbf{Supply Chain}: thanks to the blockchain applications to the thesis case, the supply chain process is
    clear and transaction-based. Moreover, the actors involved communicate in a good way keeping a well defined 
    permissioned access to the data. The management, admin side is improved and it is more transparent. It allows the 
    application to gain credibility by the end-user due to provide the proof of the worked materials in a sustainability 
    way.

    \1 \textbf{technology improvements and modular solutions}: the cross-chain solution is implemented at the 
    Application Layer. In other development, it is possible to implements a solution at a lower layer, such as 
    at the chaincode side. That solution could be more adaptable and modular than the developed one. In any case, 
    to apply that kind of solution it is mandatory the development of other modules such as a storing and mapping 
    mechanism between eth wallet and Fabric Identity.
\end{outline}



%\include{introduzione}
%\include{tecnologie}


\appendix

\backmatter
\printbibliography

\end{document}
