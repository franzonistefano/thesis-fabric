\chapter{Conclusion}

For a clearer understanding and overview of all the work produced, I listed each part developed below:

\begin{outline}
    \1 \textbf{Hyperledger Network}: the network architecture is designed based on case needs. It is 
    implemented and produced with a set of scripts that automatize all the setting up and running processes.
    \1 \textbf{Smart Contracts}: two smart contracts, that shape the use case deal, are produced.
    \1 \textbf{ERC20 Token}: an ERC20 token in running and accessible over the Ropsten network, is produced.
    \1 \textbf{Dapp}: a web-application that integrates both the blockchain used, is produced.
\end{outline}

Therefore below I listed the targets archived:

\begin{outline}
    \1 \textbf{Simple managements process}: the process is structured in a simplified way, clear and easy
    to use and manage. Each actor is associated with a specific organization inside the network, taking part 
    in the network's governance with the specific right access.

    \1 \textbf{Supply Chain}: thanks to the blockchain applications to the thesis case, the supply chain process is
    clear and transaction-based. Moreover, the actors involved communicate in a good way keeping a well defined 
    permissioned access to the data. The management, admin side is improved and it is more transparent. It allows the 
    application to gain credibility by the end-user due to provide the proof of the worked materials in a sustainability 
    way.

    \1 \textbf{technology improvements and modular solutions}: the cross-chain solution is implemented at the 
    Application Layer. In other development, it is possible to implements a solution at a lower layer, such as 
    at the chaincode side. That solution could be more adaptable and modular than the developed one. In any case, 
    to apply that kind of solution it is mandatory the development of other modules such as a storing and mapping 
    mechanism between eth wallet and Fabric Identity.
\end{outline}

