\chapter{Conclusion}

The objective of the thesis was to implement a blockchain-based solution to handle the supply chain and management processes, 
as clear as possible.
Moreover, the thesis work has to include a cross-chain part between public and private blockchain networks. In particular, 
a decentralized application has been developed that allows the users to register over that and joint to the network with 
specific rights. The cross-chain is implemented at the Application layer, developing specific API endpoints that allow 
the interoperability of the networks.  

For a clearer understanding and overview of all the work produced, I listed each part developed below:

\begin{outline}
    \1 \textbf{Hyperledger Network}: the network architecture is designed based on case needs. It is 
    implemented and produced with a set of scripts that automatize all the setting up and running processes.
    \1 \textbf{Smart Contracts}: tree smart contracts, that shape the use case deal, are produced.
    \1 \textbf{ERC20 Token}: an ERC20 token in running and accessible over the Ropsten network, is produced.
    \1 \textbf{Dapp}: a web-application that integrates both the blockchain used, is produced.
\end{outline}

In the end, the thesis archived a \textbf{simple managements process} solution, the process is structured in a simplified way, clear and 
easy to use and manage. Each actor is associated with a specific organization inside the network, taking part in the network's governance 
with the specific right access. Thanks to the blockchain applications to the thesis case, the \textbf{supply chain} process are clearer and 
transaction-based. Moreover, the actors involved communicating in a good way, keeping a well defined permissioned access to the data. The 
management, admin side, is improved and it is more transparent. 
It allows the application to gain credibility by the end-user due to provide the proof of the worked materials in a 
sustainability way. 
\bigskip

The \textbf{cross-chain} solution is implemented in the Application layer. To obtain a more modular solution is better to implement the 
cross-chain at a lower layer, such as at the chaincode side. In that way, the solution could be more adaptable and modular than the developed 
ones. In any case, to apply that kind of solution it is mandatory the development of other modules such as a storing and mapping mechanism 
between eth wallet and Fabric identity.
\bigskip

We can conclude that I archived all the fixed targets. However for \textbf{future works} the Hyperledger network could be improved, adding 
more components in order to maximize fault tolerance. The application could be improved in several parts.
The integration and the synchronization between the two networks could be handle in a better way, developing additional checks, in order to 
minimize the fault cases. The cross-chain could be implemented at a lower layer, in order to automatize mechanism such as the accounts mapping 
between the two blockchains. It makes the solution more pluggable and minimizes developing errors.  


