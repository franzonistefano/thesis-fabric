\chapter{Introduction}

Over the last few years, the Blockchain technology has been living an era of growth, with  many different applications for 
it being found in several different fields. 
\bigskip

This kind of technology is based on a peer to peer network that allows to transfer assets among users involved in the 
network, without the need for a third party. The overall network and the transactions, performed by the joint nodes, 
are handled by a consensus algorithm that performs the validation of the transactions and updates the ledger by adding 
the transactions block to the chain. All the nodes involved in the network share the same ledger that contains the 
state of the overall network. 
The most significant use of the blockchain technology is in the finance environment, which founds a great instrument 
in cryptocurrencies, exploiting the decentralized architecture and the security advantages.
\bigskip

In the last few years, many companies have been getting interested in the blockchain applications. Since, in many cases, 
the data that they share are sensitive information and companies don't want to keep them public. Therefore, 
Recently, this interest has resulted in the first permissioned blockchain solutions, maintained by a consortium 
system of nodes.
\bigskip

The blockchain world nowadays are split between permissioned and permissionless blockchains:
\begin{outline}
    \1 \textbf{Permissionless Blockchain}: The permissionless or public blockchain is a fully decentralized network where anyone in the
    world can read or send transactions, as well as participate in the consensus process.

    \1 \textbf{Permissioned Blockchain}: the permissioned solution is a blockchain where the consensus process is controlled by a 
    pre-selected set of nodes, which could be defined as a "partially decentralized" network. Moreover, a membership mechanism could 
    be implemented, in order to to handle the read and write access over the network. 
\end{outline}

Based on the above considerations, the two main challenges, faced during the thesis work will be:
\begin{outline}
    \1 \textbf{Blockchain and supply chain management}: The goal is to create a blockchain solution for the 
    management of the supply chain process of a Fashion Company. In particular, the thesis addresses the 
    issue of transparent fashion upcycling, which is generally characterized by processes involving many 
    different actors. The goal of the blockchain-based solution devised is to track the items over the flow 
    more clearly and transparently as possible.

    \1 \textbf{Cross-chain solution}: The solution implements a cross-chain interaction between Hyperledger 
    Fabric and Ethereum network. Fabric manages all the aspects related to the supply chain (orders, 
    production, part of sale process). Ethereum instead is only used to the end-user sell process of 
    the items. The goal is to reach the interoperability between public and private blockchains. 
\end{outline}

\bigskip

The target to be reached at the end of the developed thesis work would conceivably be:

\begin{outline}
    \1 \textbf{Simple management process}: The Fashion Company side, whose goal required at the end of the work is to
    reach a simplified model of the overall management process of the transactions and users involved into the system.

    \1 \textbf{Supply Chain}: A simplified handling process of the supply chain. The goal is to make as
    clear as possible the tracking process of the clothes over the actors involved, from the producing of the items
    to the selling process. 

    \1 \textbf{Technology improvements and modular solutions}: A cross-chain technology means to generalize a 
    solution that could be applied to other use cases. The integration, among more blockchains networks,
    is a big challenge nowadays.
\end{outline}